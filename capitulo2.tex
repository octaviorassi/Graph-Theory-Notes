\documentclass[APUNTE_COMPLEMENTOS.tex]{subfiles}
\begin{document}
% ------------------------------------------------------------------------------

% \chapter{Examples}

% \section{Theorem System}

% Theorem System
% The following boxes are provided:
%   Definition:     \defn 
%   Theorem:        \thm 
%   Lemma:          \lem
%   Corollary:      \cor
%   Proposition:    \prop   
%   Claim:          \clm
%   Fact:           \fact
%   Proof:          \pf
%   Example:        \ex
%   Remark:         \rmk (sentence), \rmkb (block)
% Suffix
%   r:              Allow Theorem/Definition to be referenced, e.g. thmr
%   p:              Add a short proof block for Lemma, Corollary, Proposition or Claim, e.g. lemp
%                   For theorems, use \pf for proof blocks

% --------------------------------------------------------------------------------

\chapter{Caminos, ciclos y recorridos}
\section{Definiciones y primeros resultados}
\defn{Camino}{
    Dado un grafo $G$, un camino en $G$ es una lista $v_0, e_0, v_1, \dots, e_k, v_k$ de vertices y aristas tales que $e_i = v_{i-1}, \forall{i \in [k]}$.
    Llamaremos longitud de un camino a la cantidad de aristas que posee.

    Si el grafo es simple, es suficiente con listar los vertices solamente.
}



\defn{Camino simple}{Un camino simple es un camino que no repite vertices}

\defn{Recorrido}{Un recorrido es un camino que no repite aristas}

\defn{$u,v-$camino/recorrido/camino simple}{Un $u,v-camino$ es un camino cuyos extremos son $u$ y $v$. Analogamente se define para recorridos y caminos simples}
\defn{Camino/Recorrido cerrado} {Un camino/recorrido cerrado es un $u,u-$camino/recorrido.}

\defn{Circuitos y ciclos}{Un recorrido cerrado es un circuito.
Un camino simple cerrado es un ciclo.}

\thmr{Existencia de caminos simples}{}{Todo $u,v-$camino contiene un $u,v-$camino simple.}
\pf{
Procederemos por induccion sobre la longitud del camino $n$.

$\centerdot$ Si $n = 0$, estamos en el caso de un unico vertice, luego este es un camino simple en si mismo.

$\centerdot$ Sea $n \geq 1$ y supongamos inductivamente que la propiedad vale para todo camino de longitud menor a $n$. Entonces,

$\centerdot$ Si el camino no repite vertices, no hay nada para hacer, pues ya es simple.

$\centerdot$ Supongamos que si repite (al menos) un vertice que llamaremos $w$. Entonces, si eliminamos del camino a los vertices y aristas entre dos apariciones de $w$, manteniendo una aparicion del vertice, tendremos un $u,v-$camino de longitud menor o igual a $n$. Entonces, por $HI$ este camino contiene un $u,v-$camino simple.

Luego, en particular ese es un $u,v-$camino simple del camino original, como queriamos demostrar.
}

\rmkb{
\(\centerdot\) Un circuito contiene un ciclo no trivial. Mas aun, dado un circuito y un vertice $v$, existe al menos un ciclo que esta contenido en el circuito al cual $v$ pertenece.

\(\centerdot\) Un grafo $G$ es conexo si \(\forall{u, v \in V(G)}\) existe un $u,v-$camino.

$\centerdot$ Una componente conexa de $G$ es un subgrafo conexo maximal.

$\centerdot$ El agregado de una arista reduce en a lo sumo una unidad el numero de componentes conexas del grafo.

$\centerdot$ El borrado de una arista aumenta en a lo sumo una unidad el numero de componentes conexas.

$\centerdot$ El borrado de un vertice puede aumentar el numero de componentes conexas en hasta $\lvert V(G) - 2 \rvert $ y disminuir en a lo sumo $1$.
}

\defn{Vertice y arista de corte}{
    Dado un grafo $G$, una arista de corte de $G$ es una arista cuyo borrado aumenta el numero de componentes conexas de $G$.

    Un vertice de corte es un vertice tal que $G - v$ posee mas componentes conexas que $G$.
}

\thmr{Caracterizacion de aristas de corte}{}{Una arista es de corte $\Longleftrightarrow$  no pertenece a ningun ciclo}

\pf{

$\Longleftarrow )$ Supongamos que $e\in E(G)$ no es de corte. Sea $e = uv$, $u \neq v$ y $H$ la componente conexa de $G$ a la cual pertenece $e$. Ahora, como $e$ no es de corte de $G$, tampoco puede ser de corte de $H$, luego $H/e$ debe ser conexo.

Como $u,v \in V(H/e)$, existe un $u,v-$camino simple en $H/e$. Este camino junto con $e$ forman un ciclo en $H$ y por lo tanto en $G$, absurdo.


$\Longrightarrow )$ Supongamos que $e =uv \in E(G)$ pertenece a un ciclo $C$. Sea $H$ la componente conexa tal que $e \in E(H)$. Veamos que $H/e$ es conexo.

Sean $x, y \in V(H/e)$. Como $H$ es conexo, existe un $x,y-$camino, que llamaremos $P$, en H. Entonces,

$\centerdot$ Si $e \notin {P} \Longrightarrow P$ es un $x,t-camino$ en $H/e$.

$\centerdot$ Si $e \in {P} \Longrightarrow$ consideramos el camino $P'$ que se obtiene de reemplazar la arista $e$ en $P$ por $C/e$. Como $C/e$ es un $u,v-$camino, $P'$ es un $x,y-$camino y $e \notin P'$, entonces $P'$ es un $x,y-$camino en $H/e$.

$\therefore$ $H/e$ es conexo $\Longrightarrow G/e$ tiene la misma cantidad de componentes conexas que $G \Longrightarrow$ no es de corte de $G$.

}

\rmkb{
    $\centerdot$ Si $e=uv$ es de corte, entonces $v$ es de corte?
    
    $\centerdot$ $\overline{G}-v \simeq$? $ \overline{G-v}$\

    $\centerdot$ $v\in{V(G)}$ es de corte $\Longrightarrow \overline(G) - v$ es conexo.

    $\centerdot$ Si $G$ es autocomplementario, entonces $G$ tiene un vertice de corte $\Longleftrightarrow G$ tiene un vertice de grado 1.
}
\section{Circuitos de Euler}

\defn{Grafos y circuitos eulerianos}{
    Un grafo es euleriano si posee un circuito que contiene a todas las aristas.

    Un circuito es euleriano es un circuito que recorre todas las aristas.
}

\defn{Camino maximal}{Un camino simple es maximal si no esta contenido en otro de mayor longitud.}

\lemp{}{Sea $G$ un grafo tal que $\delta (G) \geq 2$. Entonces, $G$ contiene un ciclo.}
{
    Comencemos observando que, si $G$ tiene bucles o aristas repetidas, la tesis vale trivialmente. Por lo tanto, supongamos que $G$ no tiene bucles ni aristas repetidas, y llamemos $P$ a un camino simple maximal en G. Entonces, 
    
    $\centerdot$ Si $P$ es cerrado, $\Longrightarrow P$ es un ciclo.
    
    $\centerdot$ Si $P$ no es cerrado, consideremos $u$ un extremo de $P$ cualquiera. Como $d(u) \geq 2$, existira un $v$ tal que $uv \in E(G)$ y $uv \notin P$ (pues de otra manera no seria $u$ extremo).
    Luego $v$ pertenece a P ya que de lo contrario podriamos obtener un camino simple considerando la arista $uv$ y $P$, y este nuevo camino tendria mayor longitud que $P$, lo que contradice nuestra hipotesis de $P$ maximal.
    
    $\therefore$ $v \in P$, luego $uv$ y la porcion del camino $P$ que unen $v$ con $u$ forman un ciclo.
}



\thmr{Teorema de Euler}{}{
Un grafo $G$ es euleriano $\Longleftrightarrow$ $G$ tiene a lo sumo una componente conexa no trivial y todos sus vertices tienen un grado par
}

\pf{
$\Longrightarrow)$ Supongamos que $G$ tiene un circuito euleriano $C$.

$\centerdot$ Si $E(G) = \emptyset$ la tesis es valida.

$\centerdot$ Supongamos por el contrario que $E(G) \neq \emptyset$. Cada paso de $C$ por un vertice usa 2 aristas incidentes en el, donde la primera arista y la ultima suman 2 al vertice inicial. Luego, todo vertice tiene grado par. Ademas, como 2 aristas cualesquiera de $C$ estan en el circuito euleriano, en particular estan en la misma componente conexa.

$\Longleftarrow)$ Sabemos que $d_G (v)$ es par $\forall{v \in V(G)}$ y tiene a lo sumo una componente conexa no trivial. Procederemos por induccion sobre $m = \lvert E(G) \rvert$.

$\centerdot$ $m=0$, entonces cada componente conexa es trivial luego se verifica la propiedad.

$\centerdot$ Supongamos $m \geq 1$ y que el resultado es valido para todo grafo $G$ con menos de $m$ aristas. Como tenemos al menos una arista, hay al menos una componente conexa no trivial $H$.

Observemos que $d_H (v) = d_G(v), \forall{v\in V(H)} \Longrightarrow H$ es par.

No pueden existir vertices de grado 0 ya que es conexo, entonces $\delta (H) \geq 2 \Longrightarrow H$ tiene un ciclo $C$ (lema). Sea $G' = G/E(C)$, como $C$ tiene 0 o 2 aristas incidentes en cada vertice de G, entonces cada componente conexa de $G'$ es par. Ademas, cada componente conexa tiene menos de $m$ aristas. Luego, por HI cada componente conexa de $G'$ tiene un circuito euleriano.

Ahora, para combinar estos circuitos eulerianos, comenzamos en un vertice cualquiera $v$ de $C$, recorremos $C$ y cada vez que llegamos a un vertice de alguna componente conexa, realizamos el circuito euleriano de esa componente conexa (si es que no la recorrimos antes), hasta volver al vertice de $C$ para continuar. Este recorrido culmina en $v$ y es, en efecto, un circuito euleriano.

}

\cor{
    Todo grafo par se puede descomponer en ciclos.
}

\defn{Distancia en un grafo}{
    Sea $G$ un grafo simple. Dados $u,v \in V(G)$, la distancia en $G$ entre $u$ y $v$ es la longitud del camino mas corto que los une, si existe.
    En caso contrario, decimos que la distancia es $\infty$.

    Lo notamos $dist_G (u, v) = l \in \mathbb{N}$. 
}

\rmkb{
    La maxima ${dist_G (u, v)} \forall {v, u \in V(G)}$ se denomina diametro.

    Los caminos mas cortos son simples e inducidos, esto es:
    
    Si $dist_G (u, v) = k \in \mathbb{N}$, entonces existe $P_{k+1}$ inducido en $G$.
}
    
\fact{        
    Si $diam(G) = k \in \mathbb{N}$ entonces existe un $P_{k+1}$ inducido. Mas aun, existe $P_m$ inducido $\forall{m \leq k + 1}$.
}

\cor{
    Sea $G$ un grafo simple tal que $G$ es $P_k - free$. Entonces, $diam(G) < k - 1$.
}


\prop{
    $dist_G (u, v) = 1 \Longleftrightarrow uv \in E(G)$
}

\rmkb{
Sea $G$ simple y conexo, y sea $v\in{V (G)}$, con $diam(G) \in{\mathbb{N}}$. Consideremos el conjunto $D_k (G, v) = \{ u\in{V (G)}: dist_G (u, v) = k\}$, donde $k\in{\{0, \dots, diam (G)\}}$.

Observemos que fijando un vertice $v$ y tomando todos los $k$ posibles entre $0$ y $diam (G)$ lo que obtenemos es una particion de $V (G)$.
}

\rmkb{
    Tomar una particion en $D_1, D_2, D_3, D_4$ puede ser util para probar la equivalencia entre bipartito y no tener ciclos impares.
}

\propp{
    Un camino cerrado impar contiene un ciclo impar.
}{
    Por induccion sobre la longitud del camino $l = 2k + 1, k \in \mathbb{N}$.
    
    \textbf{Caso base.} Sea $k = 1$. Entonces el unico camino cerrado posible es exactamente $C_3$, luego la propiedad vale trivialmente.

    \textbf{Hipotesis inductiva.} Supongamos que para todo $k \leq n - 1$ vale la propiedad.

    \textbf{Paso inductivo.} Consideremos el camino cerrado $P$ de longitud $2n + 1 = l$. Ahora, tenemos que este camino puede ser simple o no. Si el camino es simple, entonces por definicion es un ciclo impar, luego vale la propiedad.

    Consideremos entonces que el camino no es simple, es decir, tiene algun vertice repetido. Expresemos al camino como
    \[P = v_0 v_1 v_2 \dots v_{2n}\]
    Definamos entonces un indice $j \in \mathbb{N}$ tal que
    \[j = \min\{i \in \mathbb{N}: \exists k \in \mathbb{N}: v_i = v_k, i \neq k\}\]

    Es decir, $j$ es el primer indice de un vertice repetido en $P$. Como $j$ es repetido, podemos expresar a $P$ de la siguiente manera:
    \[P = v_0 v_1 v_2 v_3 \dots v_j v_{j+1} \dots v_k v_{k+1} \dots v_0\]

    Podemos separar entonces a $P$ en dos caminos cerrados $P_1$ y $P_2$ tales que
    \[P_1 = v_0 v_1 v_2 \dots v_j v_{k+1} \dots v_0\]
    \[P_2 = v_j v_{j+1} \dots v_{k-1} v_k\]
    \[P = P_1 + P_2 \; (P_1 \cup P_2)\]

    Observemos que, como la longitud de $P$ es impar, y es igual a la suma de las longitudes de $P_1$ y $P_2$, necesariamente alguno de los dos caminos debe tener longitud impar. Entonces, considerando ese camino, cuya longitud debera ser menor a $2k + 1$, aplicando H.I. tenemos que debe contener un ciclo impar. En particular, entonces, $P$ contiene un ciclo impar.

    $\therefore$ $P$ contiene un ciclo impar, luego por teorema de induccion vale la proposicion.


    
}

\rmkb{
    Este es un resultado mas fuerte que el anterior. Es decir, probar esta proposicion facilita la prueba de la caracterizacion de los grafos bipartitos anterior.
}

\section{Grafos dirigidos (digrafos)}

\ex{
    Si $x, y \in \mathbb{N}, x \neq y$, $x$ es un divisor maximal de $y$ si $\frac{y}{x}$ es un numero primo.

    Para $S \subseteq \mathbb{N}$, el conjunto $R = \{ (x, y) \in S^2: x$ es un divisor maximal de y $\}$.

    Para graficarlo, colocamos un vertice por cada numero de $S$ y una flecha de $x$ a $y$ si $x$ es divisior maximal de $y$.
}

\defn{Grafo dirigido (digrafo)}{
    Un grafo dirigido o digrafo es una terna que consiste en un conjunto de vertices $V(G)$, un conjunto de aristas $E(G)$ y una funcion que asigna a cada elemento de $E(G)$ un par ordenado de vertices.
}

\rmkb{
    Una notacion para un digrafo es $\vec{G}$ o $D$.

    A las aristas de un digrafo se las suele llamar tambien arcos.
}

\defn{Bucles en digrafos}{
    Un bucle en un digrafo es una arista con los mismos extremos. Aristas multiples (o repetidas) son aristas que poseen el mismo par ordenado.
}

\ex{
    Sea $f: A \longrightarrow B$ una funcion. El digrafo funcional de $f$ es el digrafo con conjunto de vertices $A$ y arcos $\{(x, f(x)): x \in A\}$.
}

\defn{Grafo subyacente}{
    El grafo subyacente de un digrafo D es el grafo G obtenido considerando los arcos como pares no ordenados.
}

\rmkb{
    Las definiciones sobre digrafos de subdigrafo o isomorfismo son analogas a las de grafos.
}

\defn{Matriz de adyacencia de un digrafo}{
    Dado $D$ un digrafo con $n$ vertices, la matriz de adyacencia de $D$ es la matriz $A(D) \in \mathbb{F}^{n \times n}$ definida por $A(D) = \{a_{ij}\}$ donde $a_{ij}$ es la cantidad de arcos de la forma $(i, j)$.   
}

\defn{Matriz de incidencia de un digrafo}{
    Dado $D$ un digrafo sin bucles, definimos a la matriz de incidencia de $D$, y la notamos $M(D)$ a la matriz tal que $M(D) = \{m_{ve}\}$ donde:

    \[ m_{ve} =
       \begin{cases} 
        1 & e = (v, u) \in E(D) \\
        -1 & e = (u, v) \in E(D) \\
        0 & e = (x, y) \in E(D) \land x \neq v \neq y \\
       \end{cases}
    \]
}

\defn{Grado de salida y de entrada de un vertice}{
    Sea $D$ un digrafo y $v \in V(G)$. Entonces,

    \begin{itemize}
    \item El grado de salida de $v$ en $G$ es la cantidad de arcos de la forma $(v, u)$ para algun $u \in V(G)$.

    \item El grado de entrada de $v$ en $G$ es la cantidad de arcos de la forma $(u, v)$ para algun $u \in V(G)$.
    \end{itemize}
    Los notaremos $d_{D}^+ (v)$ y $d_{D}^- (v)$ respectivamente.
}

\prop{
    \begin{displaymath}
        \sum_{v \in V(D)}^{} d_{D}^+ (v) = \sum_{v \in V(D)}^{} d_{D}^- (v)
    \end{displaymath}
}

\rmkb{
    Se definen de manera analoga camino, recorrido, camino simple, circuito, ciclo, respetando el orden de los arcos en la lista.
} 

\defn{Circuito euleriano dirigido}{
    Un circuito euleriano dirigido de un digrafo es un circuito que recorre cada arco exactamente una vez.
}

\lemp{Lema previo al teorema de Euler para digrafos}{
Sea $D$ un digrafo tal que $\forall{v \in V(D)}$ $d_{D}^+ (v) \geq 1$. Entonces, $D$ contiene un ciclo.
}{
Sea $\vec{P}$ un camino simple dirigido maximal en $D$ y $u \in V(D)$ el ultimo vertice de $\vec{P}$. Si $\vec{P}$ es cerrado, vale la tesis.

En caso contrario, con $d_{D}^+ (u) \geq 1$ existe un arco de la forma $(u, v)$ para algun $v \in V(D)$. Tenemos entonces dos posibilidades para este $v$.

Si $v \notin V(\vec{P})$, entonces $\vec{P}$ no seria maximal, absurdo. Luego, $v \in V(\vec{P})$, entonces $(u, v)$ junto con la parte del camino que une $v$ con $u$ forman un ciclo.
}

\thmr{Teorema de Euler}{}{
    Sea $D$ un digrafo. Entonces,

    \begin{center}
    $D$ es euleriano $\Longleftrightarrow d_{D}^+ (v) = d_{D}^- (v)$ $\forall{v \in V(D)}$ y el grafo subyacente tiene a lo sumo una componente conexa no trivial.
    \end{center}
}

\pf{
    Sigue del lema anterior, analoga al teorema de Euler para grafos.
}

\section{Ciclos de Hamilton}

\defn{Ciclo Hamiltoniano}{
    Dado un grafo $G$, un ciclo en $G$ es hamiltoniano (o de Hamilton) si contiene todos los vertices.

    En tal caso, diremos que $G$ es un $grafo \; hamiltoniano$.
}
\rmkb{
    No existe una caracterizacion sencilla de los grafos hamiltonianos como si existe de los grafos eulerianos.

    Consideraremos grafos simples, ya que los bucles y aristas repetidas son irrelevantes para la existencia de ciclos hamiltonianos.
}

\fact{
    Hay condiciones evientes que un grafo $G$ debe verificar para ser hamiltoniano:
    \begin{itemize}
        \item $\delta (G) \geq 2$
        \item $G$ conexo
        \item Las aristas con extremos en vertices de grado 2 deben aparecer en el ciclo de Hamilton (si existe).
    \end{itemize}
}

\ex{
    \begin{itemize}
        \item $K_{m,n}$ es hamiltoniano si y solo si $m = n$.
    \end{itemize}
}

\thm{\begin{center} Condicion necesaria para la existencia de ciclos hamiltonianos \end{center}}
{
    Si $G$ es hamiltoniano, entonces para todo $\emptyset \neq S \subseteq V(G)$, el grafo $G - S$ posee a lo sumo $|S|$ componentes conexas.
}
\pf{
    Supongamos $G$ es hamiltoniano. Sea $C$ un ciclo hamiltoniano en $G$ y $\emptyset \neq S \subseteq V(G)$. Sea $k \in \mathbb{N}$ el numero de componentes conexas de $G - S$.

    Observemos que la cantidad de aristas de $C$ con al menos un extremo en $S$ es a lo sumo $2 \; |S|$. 
    
    Por otro lado, para una componente conexa de $G - S$ deben existir al menos dos aristas de $C$ con un extremo en dicha componente y el otro en $S$.
    Luego, existen al menos $2k$ aristas con un extremo en S.

    Tenemos entonces que la cantidad de aristas con extremo en $S$ es al menos $2k$ y a lo sumo $2|S|$. Esto es,
    \[\therefore \; \; 2k \leq 2 |S| \Longrightarrow k \leq |S|\]
}

\cor{
    Si un grafo $G$ tiene vertices de corte entonces no es hamiltoniano.
}

\defn{Cantidad de componentes conexas}{
    Dado un grafo $G$, notaremos $c(G)$ al numero de componentes conexas de $G$.
}

\thm{
    Condicion suficiente para la existencia de ciclos hamiltonianos
}{
    Si $G$ es un grafo simple con $|V(G)| \geq 3$ tal que
    \[\forall{u, v \in V(G)}, u \neq v, uv \notin E(G) \; \; \; d_G (u) + d_G (v) \geq n \Longrightarrow G \; es \; hamiltoniano\]
}
\pf{
    Supongamos por el absurdo que existe un grafo $G$ no hamiltoniano tal que satisface las hipotesis. Notemos que si adicionamos aristas a $G$, seguira verificando las hipotesis. Por lo tanto, podemos considerar a $G$ maximal en aristas, $i. e.$ $G$ no es hamiltoniano, satisface las hipotesis, y ademas es maximal en $E(G)$, si agregamos una arista se convierte en hamiltoniano.

    Sean entonces $u, v \in V(G)$ tales que $uv \notin E(G)$ (deben existir, de otra manera seria $G \approxeq K_n \; hamiltoniano$). Tenemos que entonces en $G + uv$ existira un ciclo hamiltoniano. Ademas, ese ciclo necesariamente contendra a la arista $uv$, pues de otra manera ya existia un ciclo hamiltoniano previo a añadir la arista.

    Entonces, $G$ tiene un $u, v-camino \; simple$ que pasa por todos los vertices (el que se convierte en ciclo al agregar $uv$ a $E(G)$). Sea
    \[P = u = v_1 \; v_2 \; v_3  \; \dots v_i \; v_{i + 1} \; \dots \; v_{n - 1} \; v_n = v\]

    Queremos probar ahora que deben existir dos vertices consecutivos, $v_i , v_{i + 1}$ tales que $u$ es adyacente a $v_i$ y $v$ a $v_{i + 1}$ o viceversa. Esto nos permitiria construir un ciclo hamiltoniano en $G$.

    Sean
    \[S = \{i \in [n]: v_{i + 1} \in N(u)\}\]
    \[T = \{i \in [n]: v_i \in N(v)\}\]

    Ahora por hipotesis tenemos que $n \leq d_G (u) + d_G (v)$. Pero notemos que 

    \[ n \leq d_G (u) + d_G (v) = |S| + |T| = |S \cup T| + |S \cap T|  \]

    Observemos que $n \notin S \cap T$ y $n \notin S \cup T$, luego tenemos que el cardinal de $S \cup T$ es a lo sumo $n - 1$. En consecuencia, debera ser $|S \cap T| \geq 1$ para que se verifique la desigualdad.

    Esto es, existe un indice $i_0 \in S \cap T$, luego $v_{i_0}v \in E(G)$ y $v_{i_0 + 1}u \in E(G)$. Entonces, consideremos el ciclo
    \[C = v_1 \; \dots \; v_{i_0} \; v \; \dots \; v_{i_0 + 1} \; u\]

    $C$ es un ciclo hamiltoniano, contradiccion.

    $\therefore$ si $G$ satisface las hipotesis entonces es hamiltoniano.
}

\cor{
    Si $G$ es un grafo simple con 
    \[|V(G)| \geq 3 \; \; \; \land \; \; \; \delta (G) \geq \frac{n}{2} = \frac{|V(G)|}{2} \]

    Entonces $G$ es hamiltoniano.
}

\cor{
    Sea $G$ un grafo simple. Si existen $u, v \in V(G)$ tales que
    \[u \neq v, \; \; uv \notin E(G),\; \; d_G (u) + d_G (v) \geq n\]
    Entonces
    \[G \; es \; hamiltoniano \iff G + uv \; es \; hamiltoniano\]
}

\end{document}
