\documentclass[APUNTE_COMPLEMENTOS.tex]{subfiles}

\begin{document}

% ------------------------------------------------------------------------------

% \chapter{Examples}

% \section{Theorem System}

% Theorem System
% The following boxes are provided:
%   Definition:     \defn 
%   Theorem:        \thm 
%   Lemma:          \lem
%   Corollary:      \cor
%   Proposition:    \prop   
%   Claim:          \clm
%   Fact:           \fact
%   Proof:          \pf
%   Example:        \ex
%   Remark:         \rmk (sentence), \rmkb (block)
% Suffix
%   r:              Allow Theorem/Definition to be referenced, e.g. thmr
%   p:              Add a short proof block for Lemma, Corollary, Proposition or Claim, e.g. lemp
%                   For theorems, use \pf for proof blocks

% --------------------------------------------------------------------------------

\chapter{Flujo}

\section{Definiciones}

\defn{Red, capacidad}{
  Una red es un grafo dirigido simple ponderado que tiene un vertice distinguido $a$ con $d^a-(a) = 0$, que llamaremos origen o fuente, y un vertice distinguido $z$ con $d^+(z) = 0$, que llaraemos destino o sumidero, y tal que los pesos de las aristas son no negativos.

  Al peso de una arista $e = uv$ lo llamaremos capacidad de la arista $e$ y lo notaremos por $c(e)$ o $c(u, v)$.
}

\defn{Flujo}{
  Dada una red $G$, un flujo en $G$ es una funcion $f: E(G) \longrightarrow \mathbb{R}_0^+$. Dado un flujo $f$ en una red $G$ y $v \in V(G)$, denotaremos por $f^-(v)$ y $f^+(v)$ al flujo entrante a $v$ y saliente desde $v$ respectivamente. Es decir,
  \[f^-(v) = \sum_{u \in V(G)} f(u, v)\]
  \[f^+(v) = \sum_{u \in V(G)} f(v, u)\]
  donde $f(x, y) = 0$ si $xy \notin E(G)$.

  Diremos que $f$ es un flujo factible si para cada arista $e$ se verifica que
  \[0 \leq f(e) \leq c\]
  y para cada vertice $v \in V(G) \backslash \{a, z\}$ se verifica que
  \[f^+(v) = f^-(v)\]
  Esta ultima propiedad la llamamos conservacion del flujo.
}

\defn{Valor del flujo}{
  Dado un flujo $f$ en una red $G$, definimos al valor del flujo $f$, que denotaremos $val(f)$, como el numero $f^+(a)$. Un flujo maximo es un flujo factible de valor maximo.
}

\prop{
  Si $G$ es una red y $f$ un flujo factible de $G$, entonces $f^+(a) = f^-(z)$.
}

\section{Algoritmo de flujo maximo}

\defn{Aristas propiamente orientadas}{
  Dada una red $G$ y $u, v \in V(G)$, diremos que $uv \in E(G)$ es una arista propiamente orientada, o simplemente propia. Diremos que $uv$ es una arista impropiamente orientada, o simplemente impropia, si $vu \in E(G)$.
}

\defn{Camino $f-$aumentante.}{
  Si $f$ es un flujo factible en una red $G$, un camino $f-$aumentante es un $a,z-$camino $P$ en el grafo subyacente (es decir, ignorando las direcciones de las aristas) tal que para cada arista $e = uv$ del camino $P$, se tiene que el flujo en la arista respeta la capacidad si es una arista propia (es decir, $f(e) < c(e)$) y es no nulo si es una arista impropia (es decir, $f(e) > 0$, pues $codom(f) = \mathbb{R}_0^+$)
}

\prop{
  Si $f$ es un flujo factible en una red $G$ y $P$ es un camin $f-$aumentante con tolerancia $\Delta$, entonces la funcion
  \[f': E(G) \longrightarrow \mathbb{R}_0^+\]
  dada por
  \[f'(uv) = \begin{cases}
    f(uv) + \Delta & \text{si $uv$ es una arista propia en $P$} \\
    f(uv) - \Delta & \text{si $uv$ es una arista impropia en $P$} \\
    f(uv) & \text{en otro caso}
  \end{cases}\]
  es un flujo factible de $G$ con $val(f') = val(f) + \Delta$.
}

\defn{Algoritmo de Ford-Fulkerson}{
  \begin{itemize}
    \item \underline{Idea.} Iniciar con un flujo factible (podria ser $f(e) = 0 \; \forall e \in E(G)$). Buscamos un camino $f-aumentante$ y aumentamos el flujo en $\Delta$ unidades. Repetimos hasta que no haya caminos $f-aumentantes$. 
    
    A lo largo de la ejecucion del algoritmo iremos etiquetando cada vertice $u$ con etiquetas de la forma $(x, \alpha)$, donde diremos que $x$ es predecesor de $u$ y $\alpha = \epsilon(u)$.

    \item \underline{Entrada.} $G$ una red con fuente $a$, sumidero $z$, capacidades $c(e)$ para cada $e \in E(G)$.
    \item \underline{Salida.} Un flujo maximo para $G$.
    \item \underline{Inicializacion.} Consideramos un flujo factible $f$ de $G$, que podria ser nulo.
    \item \underline{Iteracion.}
    \subitem \textbf{Paso 1.} Etiquetar $a$ con $(-, \inf)$, y $U = \{a\}$.
    \subitem \textbf{Paso 2.} Mientras $z$ no fue etiquetado,
    Si $U = \emptyset$, entonces el algoritmo termina. Si no, elijo $v \in U$.
    
    Para todo vertice $x$ no etiquetado aun tal que $vx \in E(G)$, si $f(vx) < c(vx)$, entonces agrego $x$ a $U$ y lo etiqueto con 
    \[(v, \; \min\{val(v), \; c(vx) - f(vx)\})  \]

    Para todo vertice $x$ no etiquetado aun tal que $xv \in E(G)$, si $f(xv) > 0$ entonces agrego $x$ a $U$ y lo etiqueto con
    \[(v, \; \min\{val(v), \; f(vx)\})  \]

    A continuacion, borramos a $v$ de $U$.

    \subitem \textbf{Paso 3.} Si $z$ esta etiquetado, sea $\Delta = val(z)$. Construyo un $a, z-$camino $P$ de ``atras hacia adelante'' de la siguiente manera: $w_0 = z$ y para $i > 0$, definimos a $w_i$ como el predecesor de $w_{i-1}$, hasta llegar a $w_k = a$. De esta manera, $w_k , w_{k-1}, \dots, w_0$ es un $a,z-$camino. Para cada arista $e = w_i w_{i-1}$ de $P$, actualizamos el flujo $f$.

    Si $e$ es una arista propia, $f(e) \longleftarrow f(e) + \Delta$, y si $e$ es una arista impropia, $f(e) \longleftarrow f(e) - \Delta$.

    A continuacion, borramos todas las etiquetas y volvemos al paso 1.
  \end{itemize}
}

\section{Flujos maximos y cortes minimos}

\defn{Corte}{En una red $G$ con una fuente $a$ y sumidero $z$, un $corte$ $(S, T)$ consiste en el conjunto de aristas

\[\{uv \in E(G): u \in S, \; v \in T\}\]
donde $S$ y $T$ son conjuntos de vertices que forman una particion de $V(G)$ tales que $a \in S$, $z \in T$.

La capacidad del corte $(S, T)$, que denotaremos $cap(S, T)$, es la suma de las capacidades de las aristas en el corte. Un corte minimo es un corte con capacidad minima.

}

\rmkb{
  Notemos que una red es un grafo dirigido, luego un corte $(S, T)$ tiene solamente aquellas aristas que van desde un vertice $u \in S$ hacia un vertice $v \in T$ (no tiene por que cubrir todas las aristas, solo particiona los vertices).

  Ademas, si $(S, T)$ es un corte, entonces $T = \overline{S}$, ya que son una particion. Luego, al referirnos a un corte, basta con indicar el conjunto $S$, y a veces notaremos directamente $(S, \overline{S})$.
}

\prop{
  Sea $f$ un flujo factible en una red $G$, y sea $(S, \overline{S})$ un corte de $G$. Entonces, 
  \[val(f \leq cap(S, \overline(S)))\]
}

\cor{
  Sea $G$ una red. Si $f$ es un flujo y $(S, \overline{S})$ un corte de $G$ tales que $val(f) = cap(S, \overline{S})$, entonces $f$ es un flujo maximo, $(S, \overline{S})$ es un corte minimo.

  Mas aun, la igualdad $val(f) = cap(S, \overline{S})$ vale si y solo si para toda arista $uv$ con $u \in S$, $v \notin S$, $f(u, v) = c(u, v)$ y para toda arista $uv$ con $u \notin S$ y $v \in S$, $f(u, v) = 0$.
}

\thm{}{
  En toda red, el valor de un flujo maximo es igual a la capacidad de un corte minimo.
}

\cor{
  Si todas las capacidades en una red son numeros naturales, entonces existe un flujo maximo que le asigna un flujo entero a cada arista.
}

\end{document}